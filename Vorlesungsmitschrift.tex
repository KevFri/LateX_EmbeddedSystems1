\section{Mitschrift aus Vorlesung}
\subsection{Vorlesung 04.04.2017 - IO-Bibliothek}

\begin{itemize}%Aufzählung mit Numerierung
		\item jeder uC is universell aufgebaut und muss für jeden Fall individuell konfiguriert werden
		\item kleine Bibliothek zum anpassen der IO-Ports ist praktisch
		\item Lese und Schreib Funktionalität soll realisiert werden
		\item Datenblatt Figure 11-1 -> Pins müssen über Treiber realisiert werden, Schmitt-Trigger ist enthalten
		\item Pegel um Treiber zu aktivieren? nachlesen Schaltbild könnte falsch sein
		\item Lesezugriff über Port, Schreibezugriff über Latch
		\item Erkenntnis: Lesen des Datenblattes liefert Aufschluss wie die Pins zu beschalten sind. 
		
\end{itemize}

\subsection{Vorlesung 11.04.2017 - Timer-Konfiguration}

\begin{itemize}%Aufzählung mit Numerierung
	\item An beliebigen Pin soll PWM ausgegeben werden-> Nachbildung durch Software
	\item Software PWM soll Leuchtdiode ansteuern
	\item Der dsPIC22EP810MU verfügt über 9 Zeitgeber Timer1 ... Timer9
	\item Unterschiedliche Struktur 16/32Bit
	\item Können Interrupts auslesen
	\item Übersicht/Kurzfassung in Datenblatt Abschnitt 12 / S271ff.
	\item Dataillierte Information Mikrochip dsPIC33E/PIC24E Familiy Reference Manual Abschnitt 11, mit Konfigurationscode, sehr hilfreich
	\item Timer1: Figure 12-1 16-Bit Timer1 Modul Block Diagramm
	\item Timer ist im Prinzip Zähler -> 16Bit Register das incrementiert werden kann
	\item Prescaler Teilerfaktor kann eingestellt werden
	\item Abbruch durch Vergleicher -> PR1. 
	\item Flag kann gesetzt werden -> Interrupt erzeugen und/oder Zähler zurücksetzen
	\item es gibt viele Möglichkeiten, die durch äußere Parameter einstellbar sind
	\item 16-Bit-Timer-1 verwendbar als
	\begin{itemize}
		\item Timer (Zeitgeber), auch in Verbindung mit dem low Power 32kHz Quarzoszillator -> kann Rechner z.B. schlafen legen
		\item Gated Timer (Periodendauermessung eines externen Signals)
		\item Counter (Zählen externer Impulse)
	\end{itemize}
	\item Initialation Coder for 16-bit Timer Module Example 11-1, löst auch Interrupts aus, benötigen wir (noch) nicht. Abschnitt 11, S.11
	\item Alternativen zur Konfiguration mittels Handcodierung
	\begin{itemize}
		\item OpenTimerX-Funktionen aus der plib-Bibliothek
		\item MPLAB Code Configurator (grafisches Tool) ... leider nicht für alle µC verfügbar, für unseren auch nicht
	\end{itemize}
	\item Timer3 Konfiguration
	\begin{itemize}
		\item TypeC Timer kann auch A/D Wandlungen zyklisch abtasten
		\item Timer sind kaskadierbar
	\end{itemize}
\end{itemize}

\subsection{Vorlesung 25.04.2017 - SystemTime}

\begin{itemize}%Aufzählung mit Numerierung
	\item Millis soll in 32Bit Variable gespeichert werden
	\item Interrupt unterbricht laufende Operationen und verzweigt an eine Stelle im Programmcode
	\item Interrupt soll ms Tigger incrementieren
	\item 16Bit rechner \& 32Bit Variable kann Probleme beim lesen verursachen, da mehrere Operationen verwendet werden
	\item Timer 3 in Sekundentakt
\end{itemize}
Ein Interrupt ist die Unterbrechung eines Programm durch ein Ereignis, z.B.:
\begin{itemize}
	\item Signale an I/O Pins
	\item Zeitgeber (Timer)
	\item ADC mit Konvertierung fertig
	\item Serielle Kommunikation
	\item EEPROM-Schreibzugriff beendet
\end{itemize}
Das Vorhandensein von Interruptquellen ist abhängig von im Baustein implementierten Peripheriekomponenten.\newline\newline
Interrupt:
\begin{itemize}
	\item Ein Interrupt unterbricht das laufende Programm und verzweigt zur Interruptserviceroutine ISR
	\item Nach Abarbeiten der ISR wurd das unterbrochene Programm fortgesetzt
\end{itemize}
%
Interruptvektor:
\begin{itemize}
	\item Sammlung aller Programmspeicherstellen bei denen Interrupts stehen
	\item ...
\end{itemize}
%
Interruptsserviceroutine
\begin{itemize}
	\item Code, der nach einer Interruptanforderung abgearbeitet wird
\end{itemize}
%
Interruptlatenzzeit
\begin{itemize}
	\item Zeitraum zwischen Interruptanforderung und Abarbeitung des Befehls in der ISR
\end{itemize}
%
Trap:
\begin{itemize}
	\item Spezielle Interrupt, der bei schwerwiegenden Fehlern ausgelöst wird (z.B. Division durch Null, Oszillatorausfall etc.)
	\item führt Standartmäßig zu einem Reset
\end{itemize}
%
2Schritte:
\begin{itemize}
	\item Konfigurieren der Hardware, die den Interrupt auslösen soll
	\item Konfiguration des dazugehörigen Interrupts
\end{itemize}
%
Interrupts konfigurieren:
\begin{itemize}
	\item IECx: Interrupt enable
	\item 	
\end{itemize}
%
ISR vs. C-Funktion
\begin{itemize}
	\item Da nicht vorhergesagt werden kann, an welcher Stelle ein Interrupt den Programmablauf unterbricht, müssen funktionskritische Daten z.B.: ALU-Stati, von der ISR gesichert sein und vor der Rückkehr zum unterbrochenen Programm wiederhergestellt werden
	\item Das dazugehörige Code wird vom Compiler eingefügt
	\item Der C-Compiler benötigt eine zusätzliche Information darüber, dass eine Codesequenz eine ISR und keine gewöhnliche Funktion ist (spezielle Syntax)	
\end{itemize}

%
ISR:
\begin{itemize}
	\item Unterscheidung von einer Funktion durch Schlüsselwörter oder pragma-Direktiven
	\item ist vom Typ void
	\item keine Parameterübergabe
	\item soll kurze Laufzeiten haben
	\item soll keine anderen Funktionen aufrufen
	\item Globale Variablen, die in der ISR Routine verändert werden als volatile deklarieren
	\item Nach Abarbeitung des Interrupts Interrupt Flag-löschen
	\item Flag von Timer zurücksetzen
\end{itemize}

\subsection{Vorlesung 02.05.2017 - Non-blocking Code}
\begin{itemize}
	\item Implementierung mit State Machines...
\end{itemize}

\subsection{Vorlesung 16.05.2017 - LCD-Bibliothek}
\begin{itemize}
	\item LC-Display
	\begin{itemize}
		\item FDCC1602L-NSWBBW-91LE
		\item alphanumerisches Display
		\item 2Zeilen, 16Zeichen/Zeile
		\item HD44780 Controller, steuer Display Hardware an
		\item LC-Controller ist über einen 8bit Datenstrom angeschlossen
		\item Die Kommunikation zwischen PIC und HD44780 erfolgt nach einem definierten Protokoll
		\item Initialisierungssequenz für HD44780 Controller notwendig
		\item Blockschaltbild von Display-Controller
	\end{itemize}
	\item Aus Sicht des uC-Programmierers
	\begin{itemize}
		\item HD44780 hat ein Instruktions-(IR) und ein Datenregister DR
		\item beide Register sind 8Bit Register
		\item Über diese beiden Register erfolgt die Kommunikation mit dem PIC uC
		\item Interne Register
		\begin{itemize}
			\item DDRAM: Display Data RAM
			\item CGRAM: Character Generator RAM
		\end{itemize}			
		\item 11 ansteuerbare Pins, D0-D7, E(nable), RS,RW
		\begin{itemize}
			\item RS Register Select 0: Incstruction, 1: Data
			\item R/notW 0:Write, 1:Read
			\item E Enable
			\item Dx D7...D0 (8-Bit-Modus) D7:BusyFlag
			\item 8-Bit-Daten werden in einem Paket übertragen
			\item mit fallender Flanke von E werden die Daten übernommen
			\item Achtung bei Schreib/Lese Zugriff muss Pin Mode umgeschalten werden
			\item Initialisierungssequenz: HD44780-Datenblatt S.45 mit Zeitangaben(!)
		\end{itemize}
		\item busy flag sollte abgefragt werden vor Schreibzugriff
		\item Funktionen
		\begin{itemize}
			\item configMyLCDport();
			\item clockLCDenable();
			\item initMyLCD();
			\item sendCommandLCD(uint8\_t u8Data);
			\item writeDataLCD(uint8\_t u8data);
			\item setDDRAMAdressLCD(uint8\_t u8adress)
			\item uint8\_t readBusyFlagAndAdressLCD();
			\item void putsLCD(char *pData);
			\item void putCharLCD(char c);
		\end{itemize}
		\item Anpassung an die Erfordernisse des kooperativen Multitasikings
		\begin{itemize}
			\item Delays >10us durch State-Machine ersetzen
			\item Ansatz: putc() schreibt ein Zeichen auf das LCD
			\item isReadyLCD() fragt das Busy-Flag ab und gibt true zurück, wenn der nächste character geschrieben werden kann
		\end{itemize}
			\item Informationen
		\begin{itemize}
			\item Datenblatt FDCC1602L-NSWBBW-91LE z.B. farnell
			\item Datenblatt Dispaycontroller sparkfun, maritex
			\item Beschreibungen wikipedia HD44780, \url{sprut.de/electronic/lcd/index.htm}
			\item Simulator
		\end{itemize}
\end{itemize}
\end{itemize}

\subsection{Vorlesung 16.06.2017 - AD-Wandler}
\begin{itemize}
	\item Two Modules:
	\begin{itemize}
		\item ADC1: Up to 32 analog input channels
		\item ADC2: Up to 16 analog input channels
	\end{itemize}
	\item Key Features
	\begin{itemize}
		\item 10- or 12-bit ADC
		\item Successive Approximation (SAR) Conversion (Dt.: Wägeverfahren)
		\subitem Kompromiss zw. Geschwindigkeit und Aufwand
		\item Conversion Speed of up to 1,1Msps (10bit) or 500ksps(12Bit)
		\item simultaneous sampling of up to 4 analog input pins (10bit)
	\end{itemize}
	\item ADCx MODULE BLOCK DIAGRAM
	\begin{itemize}
		\item 4 Sample and Hold -Glieder
		\item Multiplexer
		\item Referenzspannungen 
		\subitem $\rightarrow$ Wir verwenden interne Referenzen
	\end{itemize}
	\item AD-Umsetzer nach dem Wägeverfahren
	\subitem siehe Abb 17.41 Tietze, Schenk, Gamm
	\item Flash-Converter sind sehr schnell, jedoch sehr aufwändig zu realisieren
	\item EDA dsPICC33-Board
	\begin{itemize}
		\item AN0, AN1
		\item Potis mit SW500 zuschaltbar
		\item Analoge Eingänge auf J501 herausgeführt
	\end{itemize}
	\item
\end{itemize}

\begin{itemize}
	\item 
	\item
	\item
\end{itemize}
