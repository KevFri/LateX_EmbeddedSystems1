\section{LCD-Bibliothek}
\subsection{Aufgabenstellung}
Schreiben Sie eine Bibliothek zur Ansteuerung des LC-Displays auf dem EDAdsPIC33-Board:
\begin{itemize}
	\item Initialisierung (blocking Code erlaubt)
	\item Zyklisches Kopieren eines Schattenspeichers in den Display-Speicher\newline
			(non-blocking code).
\end{itemize}

\subsubsection*{Hinweise zur Implementierung}

Literatur:
\begin{itemize}
	\item DiJasio(2012): Chapter 9\newline
			\url{https://www.dropbox.com/sh/4bwfkqux88a3j62/AAAhId8mi25vsYaToLJbrS1Ba?dl=0\&preview=09_Glass_Bliss.pdf}
	\item Datenblatt FDCC1602L-NSWBBW-91LE, \newline \url{http://farnell.com/datasheets/653662.pdf}
	\item Datenblatt Displaycontroller \newline
			\url{https://sparkfun.com/datasheets/LCD/HD44780.pdf}\newline
			\url{http://download.maritex.com.pl/pdfs/op/TC1602E-06H.pdf}
	\item Simulator \newline
	 \url{http://dinceraydin.com/djlcdsim/djlcdsim.html}
	
\end{itemize}

Vorgehensweise:
\begin{itemize}
	\item Schreiben Sie zunächst eine Funktion putLCD(), die ein Zeichen auf dem Display ausgibt (blocking code)
	\item Ändern Sie die Funktion so, dass sie kooperativ wird.
	\item Implementieren Sie anschließend das Kopieren des Schattenspeichers
\end{itemize}

\subsection{Lösung}

\begin{enumerate}
		\item Lösungsansatz 1
		\item Lösungsansatz 2
		\item Lösungsansatz 3
		\item Lösungsansatz 4
\end{enumerate}


%Beispiel für Quellcode Listening
\begin{lstlisting}[frame=htrbl, caption={Listening Bezeichnung}, label={lst:Referenzname}]
//Quellcode
\end{lstlisting}

