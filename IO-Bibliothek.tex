\section{IO-Bibliothek}
\subsection{Aufgabenstellung}
Die populäre Arduino-Plattform (\url{https://www.arduino.cc/en/Reference/}) 
kapselt die Pinkonfiguration und -ansteuerung mit folgenden Funktionen:
\begin{itemize}
\item pinMode()
\item digitalRead()
\item digitalWrite()
\end{itemize}
Übertragen Sie dieses Konzept auf das EDA-Board. Anwendungsbeispiele:
\begin{itemize}
\item pinMode(SW1, IPUT\_PULLUP)
\end{itemize}
soll den Pin, an den SW1 angeschlossen ist, als digitalen Eingang konfigurieren und den Pullup-Widerstand einschalten.
\begin{itemize}
	\item digitalWrite(LED203, HIGH)
\end{itemize}
soll an dem Pin, an den LED203 angeschlossen ist,
einen High-Pegel ausgeben.
Verwenden Sie die Bezeichner aus dem Schaltplan.
Modularisieren Sie Ihre Software, verwenden Sie dazu die Dateinamen \textit{edaPIC33Hardware.h} und \textit{edaPIC33Hardware.c}.\newline \newline
Dokumentieren Sie die Funktionen mit Doxygen.\newline \newline
Messen Sie die Zeit, die zur Ansteuerung eines Ausgangspins mit den IO-Bibliotheksfunktionen notwendig ist
und vergleichen Sie diese mit einem direkten Schreiben in die entsprechenden Hardwareregister.

\subsection{Lösung}
Alle Device-Pins (ausgenommen VDD, VSS, MCLR and OSC1/CLKI) sind aufgeteilt auf die Ports für Peripheriegeräte und parallel I/O Ports. Alle I/O Ports sind Schmitt-Trigger Input (verbesserte Störungsunempfindlichkeit / Rauschempfindlichkeit).\newline
Alle Port Pins besitzen acht Register, durch diese Register lässt sich der I/O Port wie in Tabelle \ref{tab:ioports} dargestellt konfigurieren. Die Register Maps sind in Abbildung \ref{image:PortMap} dargestellt.\newline\newline
Ein Beispiel wie auf die einzelnen Register Bits zugegriffen werden kann und wie ein Port konfiguriert werden kann ist in Listing \ref{lst:confregport} zu sehen.\newline\newline
Die relevanten Auszüge aus dem Datenblatt sind in Abbildung \ref{image:PortMap} abgebildet.\newline\newline
Es wurde die Bibliothek mit den Dateien \textit{edaPIC33Hardware.h} und \textit{edaPIC33Hardware.c} erstellt. Die Funktion wurde mit Doxygen dokumentiert (TODO Anhang...).

\newpage
\begin{table}
\begin{tabular}{|l|l|}
	\hline 
	\textbf{Register Bit} & \textbf{Function}\\ 
	\hline
	TRISx 	&determines whether the pin is an input 	or an output\\
			&0:=output, 1:=input\\
			&default: all port pins are defined as inputs after a reset\\
	\hline 
	PORTx 	&read reads the port, write writes the latch \\ 
	\hline 
	LATx 	&read reads the latch, write writes the latch \\ 
	\hline 
	ODCx 	&configures pin for digital or open-drain output \\
			&0:=digital output, 1:=open drain output\\
	\hline 
	CNENx 	&enables change notification (CN) interrupts\\
			&0:=interrupts disabled, 1:=interrupts enabled\\
	\hline 
	CNPUx 	&enables weak pullup\\
			&0:=pullup disabled, 1:=pullup enabled\\  
	\hline 
	CNPDx 	&enables weak pulldown\\
			&0:=pulldown disabled, 1:=pulldown enabled\\  
	\hline 
	ANSELx 	&controls the operation of the analog port pins\\
			&0:=port operates as digital I/O port\\
			&1:=port operates as analog I/O port\\
	\hline
	note 	&Any bit and its associated data and control registers that are not\\
			&valid for a particular device is disabled. This means the correspon\\
			&-ding LATx and TRISx registers and the port pin are read as zeros.\\
	\hline
	note	&The open-drain feature allows the generation of outputs higher\\
			&than VDD (e.g., 5V on a 5V tolerant pin) by using external pull-up\\
			&resistors. The maximum open-drain voltage allowed is the same as \\
			&the maximum VIH specification for that pin.\\
	\hline
	note	&Pull-ups and pull-downs on change notification (CN) pins should\\
			&be disabled when the port pin is configured as a digital output.\\
	\hline
\end{tabular}
\caption{Konfigurationsmöglichkeiten der I/O Ports}
\label{tab:ioports}
\end{table}

\begin{lstlisting}[frame=htrbl, caption={Konfigurieren des Ports RB8 als Digitaler Input mit Pullup Widerstand}, label={lst:confregport}]
//configure RB8 as digital input with pullup
TRISBbits.TRISB8=1;     //configure as input
ANSELBbits.ANSB8=0;     //configure as digital
CNENBbits.CNIEB8=0;     //disable change notification interrupt
CNPUBbits.CNPUB8=1;     //enable weak pullup
CNPDBbits.CNPDB8=0;     //disable weak pulldown
\end{lstlisting}
\newpage
\subsection{Update der I/O Bibliothek}
Im laufe der Vorlesung stellte es sich als sehr mühsam und fehleranfällig heraus jeden Port des PIC von Hand wie in Listing \ref{lst:confregport} zu konfigurieren.\newline
Deshalb wurde ein Konzept entwickelt, mit welchem man möglichst allgemein und mit wenig Programmieraufwand jeden einzelnen I/O Port konfigurieren kann. \newline

\begin{lstlisting}[frame=htrbl, caption={Funktionsprototypen edaPIC33Hardware-Library}, label={lst:edaPIC33Hardware}]
void pinMode(const uint8_t ui8Port,const  uint8_t ui8Mode);

void digitalWrite(const uint8_t ui8Port,const  uint8_t ui8Value);

void digitalToggle(const uint8_t ui8Port);

uint8_t digitalRead(const uint8_t ui8Port);

uint16_t* getpTRIS(uint8_t Port);

uint16_t* getpPORT(uint8_t Port);

uint16_t* getpLAT(uint8_t Port);

uint16_t* getpODC(uint8_t Port);

uint16_t* getpCNEN(uint8_t Port);

uint16_t* getpCNPU(uint8_t Port);

uint16_t* getpCNPD(uint8_t Port);

uint16_t* getpANSEL(uint8_t Port);

void setBit(uint16_t* pui16Var, uint8_t ui8Bit, uint8_t ui8Value);

uint8_t getBit(uint16_t ui16Var, uint8_t ui8Bit);

uint8_t getPortBitNumb(uint8_t Port);

\end{lstlisting} 

\newpage
\begin{lstlisting}[frame=htrbl, caption={pinMode() Funktion}, label={lst:pinmode}]
//Supported Modes:DIGITAL_INPUT, DIGITAL_INPUT_PULLDOWN, DIGITAL_INPUT_PULLUP, DIGITAL_OUTPUT, OPEN_DRAIN_OUTPUT, ANALOG_INPUT, ANALOG_OUTPUT, ANALOG_INPUT_PULLDOWN, ANALOG_INPUT_PULLUP
void pinMode(const uint8_t ui8Port,const uint8_t ui8Mode)
{
uint16_t* pTRIS =  getpTRIS(ui8Port);    //Output/Input select
uint16_t* pCNEN =  getpCNEN(ui8Port);    //CN Interrupt Enable Bit
uint16_t* pCNPU =  getpCNPU(ui8Port);    //Pull-Up enable Bit
uint16_t* pCNPD =  getpCNPD(ui8Port);    //Pull-Down enable Bit
uint16_t* pANSEL = getpANSEL(ui8Port);   //Analog select Bit
uint16_t* pODC =   getpODC(ui8Port);     //open drain select Bit
switch(ui8Mode)
{   //set ANSEL, TRIS, CNEN, CNPU, CNPD and ODC bit
	case DIGITAL_INPUT:
		setBit(pANSEL,getPortBitNumb(ui8Port), 0);
		setBit(pTRIS, getPortBitNumb(ui8Port), 1);      
		setBit(pCNEN, getPortBitNumb(ui8Port), 0);
		setBit(pCNPU, getPortBitNumb(ui8Port), 0);
		setBit(pCNPD, getPortBitNumb(ui8Port), 0);
		setBit(pODC,  getPortBitNumb(ui8Port), 0);
	break;
	
	case DIGITAL_INPUT_PULLUP:
		setBit(pANSEL,getPortBitNumb(ui8Port), 0);
		setBit(pTRIS, getPortBitNumb(ui8Port), 1);      
		setBit(pCNEN, getPortBitNumb(ui8Port), 0);
		setBit(pCNPU, getPortBitNumb(ui8Port), 1);
		setBit(pCNPD, getPortBitNumb(ui8Port), 0);
		setBit(pODC,  getPortBitNumb(ui8Port), 0);
	break;
	
	case DIGITAL_INPUT_PULLDOWN:
		setBit(pANSEL,getPortBitNumb(ui8Port), 0);
		setBit(pTRIS, getPortBitNumb(ui8Port), 1);      
		setBit(pCNEN, getPortBitNumb(ui8Port), 0);
		setBit(pCNPU, getPortBitNumb(ui8Port), 0);
		setBit(pCNPD, getPortBitNumb(ui8Port), 1);
		setBit(pODC,  getPortBitNumb(ui8Port), 0);
	break;
	
	case DIGITAL_OUTPUT:
		setBit(pANSEL,getPortBitNumb(ui8Port), 0);
		setBit(pTRIS, getPortBitNumb(ui8Port), 0);      
		setBit(pCNEN, getPortBitNumb(ui8Port), 0);
		setBit(pCNPU, getPortBitNumb(ui8Port), 0);
		setBit(pCNPD, getPortBitNumb(ui8Port), 0);
		setBit(pODC,  getPortBitNumb(ui8Port), 0);
	break;
	
	case OPEN_DRAIN_OUTPUT:
		setBit(pANSEL,getPortBitNumb(ui8Port), 0);
		setBit(pTRIS, getPortBitNumb(ui8Port), 0);      
		setBit(pCNEN, getPortBitNumb(ui8Port), 0);
		setBit(pCNPU, getPortBitNumb(ui8Port), 0);
		setBit(pCNPD, getPortBitNumb(ui8Port), 0);
		setBit(pODC,  getPortBitNumb(ui8Port), 1);
	break;
	
	case ANALOG_OUTPUT:
		setBit(pANSEL,getPortBitNumb(ui8Port), 1);
		setBit(pTRIS, getPortBitNumb(ui8Port), 0);      
		setBit(pCNEN, getPortBitNumb(ui8Port), 0);
		setBit(pCNPU, getPortBitNumb(ui8Port), 0);
		setBit(pCNPD, getPortBitNumb(ui8Port), 0);
		setBit(pODC,  getPortBitNumb(ui8Port), 0);
	break;
	
	case ANALOG_INPUT:
		setBit(pANSEL,getPortBitNumb(ui8Port), 1);
		setBit(pTRIS, getPortBitNumb(ui8Port), 1);      
		setBit(pCNEN, getPortBitNumb(ui8Port), 0);
		setBit(pCNPU, getPortBitNumb(ui8Port), 0);
		setBit(pCNPD, getPortBitNumb(ui8Port), 0);
		setBit(pODC,  getPortBitNumb(ui8Port), 0);
	break;
	
	case ANALOG_INPUT_PULLDOWN:
		setBit(pANSEL,getPortBitNumb(ui8Port), 1);
		setBit(pTRIS, getPortBitNumb(ui8Port), 1);      
		setBit(pCNEN, getPortBitNumb(ui8Port), 0);
		setBit(pCNPU, getPortBitNumb(ui8Port), 0);
		setBit(pCNPD, getPortBitNumb(ui8Port), 1);
		setBit(pODC,  getPortBitNumb(ui8Port), 0);
	break;   
	
	case ANALOG_INPUT_PULLUP:
		setBit(pANSEL,getPortBitNumb(ui8Port), 1);
		setBit(pTRIS, getPortBitNumb(ui8Port), 1);      
		setBit(pCNEN, getPortBitNumb(ui8Port), 0);
		setBit(pCNPU, getPortBitNumb(ui8Port), 1);
		setBit(pCNPD, getPortBitNumb(ui8Port), 0);
		setBit(pODC,  getPortBitNumb(ui8Port), 0);
	break;
	}
}
\end{lstlisting}

% Vorlage für Quellcode Listing	
\begin{lstlisting}[frame=htrbl, caption={digitalWrite() Funktion}, label={lst:digitalwrite}]
void digitalWrite(const uint8_t ui8Port,const uint8_t ui8Value)
{  
	if(ui8Value)
		setBit(getpLAT(ui8Port), getPortBitNumb(ui8Port), 1);
	else
		setBit(getpLAT(ui8Port), getPortBitNumb(ui8Port), 0);
}
\end{lstlisting}

\begin{lstlisting}[frame=htrbl, caption={digitalToggle() Funktion}, label={lst:digitalToggle}]
void digitalToggle(const uint8_t ui8Port)
{
	digitalWrite(ui8Port, !digitalRead(ui8Port));
}
\end{lstlisting}

\begin{lstlisting}[frame=htrbl, caption={digitalRead() Funktion}, label={lst:digitalRead}]
uint8_t digitalRead(const uint8_t ui8Port)
{
	return getBit(*getpPORT(ui8Port), getPortBitNumb(ui8Port));
}
\end{lstlisting}

\begin{lstlisting}[frame=htrbl, caption={getPortBitNumb() Funktion}, label={lst:getPortBitNumb}]
uint8_t getPortBitNumb(uint8_t Port)
{
	switch(Port)
	{
		case RA0:  return 0;
		case RA1:  return 1;
		case RA2:  return 2;
		//[...] fuer alle Pins definiert
		case RB0:  return 0;
		case RB1:  return 1;
		case RB2:  return 2;
		//[...]
		case RG13: return 13;
		case RG14: return 14;
		case RG15: return 15;       
	}
	return 0;
}
\end{lstlisting}


\begin{lstlisting}[frame=htrbl, caption={setBit() Funktion}, label={lst:setBit}]
void setBit(uint16_t* pui16Var, uint8_t ui8Bit, uint8_t ui8Value)
{
if(ui8Bit < 16 && ui8Value<2)
{
	if(ui8Value)
		*pui16Var |= (0x0001 << ui8Bit); //set to 1
	else
		*pui16Var &= (~(0x0001 << ui8Bit)); //set to 0
	}
}
\end{lstlisting}

\begin{lstlisting}[frame=htrbl, caption={getBit() Funktion}, label={lst:getBit}]
uint8_t getBit(uint16_t ui16Var, uint8_t ui8Bit)
{
return (ui16Var & ( 1 << ui8Bit )) >> ui8Bit;
}
\end{lstlisting}

\begin{lstlisting}[frame=htrbl, caption={getpTRIS() Funktion}, label={lst:getpTRIS}]
uint16_t* getpTRIS(uint8_t Port)
{
	if( Port == RA0 ||  Port == RA1 ||  Port == RA2 ||  Port == RA3 ||  Port == RA4 ||  Port == RA5 ||  Port == RA6 ||  Port == RA7 ||  Port == RA9 ||  Port == RA10 || Port == RA14 || Port == RA15)
	return (uint16_t*)&TRISA;
	//[...] fuer alle Pins definiert
	if( Port == RG0 ||  Port == RG1 ||  Port == RG2 ||  Port == RG3 ||  Port == RG6 ||  Port == RG7 ||  Port == RG8 ||  Port == RG9 ||  Port == RG12 || Port == RG13 || Port == RG14 || Port == RG15)
	return (uint16_t*)&TRISG;
	return 0;
}
\end{lstlisting}