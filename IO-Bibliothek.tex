\section{IO-Bibliothek}
\subsection{Aufgabenstellung}
Die populäre Arduino-Plattform (\url{https://www.arduino.cc/en/Reference/}) 
kapselt die Pinkonfiguration und -ansteuerung mit folgenden Funktionen:
\begin{itemize}
\item pinMode()
\item digitalRead()
\item digitalWrite()
\end{itemize}
Übertragen Sie dieses Konzept auf das EDA-Board. Anwendungsbeispiele:
\begin{itemize}
\item pinMode(SW1, IPUT\_PULLUP)
\end{itemize}
soll den Pin, an den SW1 angeschlossen ist, als digitalen Eingang konfigurieren und den Pullup-Widerstand einschalten.
\begin{itemize}
	\item digitalWrite(LED203, HIGH)
\end{itemize}
soll an dem Pin, an den LED203 angeschlossen ist,
einen High-Pegel ausgeben.
Verwenden Sie die Bezeichner aus dem Schaltplan.
Modularisieren Sie Ihre Software, verwenden Sie dazu die Dateinamen \textit{edaPIC33Hardware.h} und \textit{edaPIC33Hardware.c}.\newline \newline
Dokumentieren Sie die Funktionen mit Doxygen.\newline \newline
Messen Sie die Zeit, die zur Ansteuerung eines Ausgangspins mit den IO-Bibliotheksfunktionen notwendig ist
und vergleichen Sie diese mit einem direkten Schreiben in die entsprechenden Hardwareregister.

\subsection{Lösung}
Alle Device-Pins (ausgenommen VDD, VSS, MCLR and OSC1/CLKI) sind aufgeteilt auf die Ports für Peripheriegeräte und parallel I/O Ports. Alle I/O Ports sind Schmitt-Trigger Input (verbesserte Störungsunempfindlichkeit / Rauschempfindlichkeit).\newline
Alle Port Pins besitzen acht Register, durch diese Register lässt sich der I/O Port wie in Tabelle \ref{tab:ioports} dargestellt konfigurieren. Die Register Maps sind in Abbildung \ref{image:PORTA}-\ref{image:PORTG} dargestellt.\newline\newline
Ein Beispiel wie auf die einzelnen Register Bits zugegriffen werden kann und wie ein Port konfiguriert werden kann ist in Listing \ref{lst:confregport} zu sehen.\newline\newline
Die relevanten Auszüge aus dem Datenblatt sind in den Abbildungen \ref{image:page207}-\ref{image:page209} abgebildet.\newline\newline
Es wurde die Bibliothek mit den Dateien \textit{edaPIC33Hardware.h} und \textit{edaPIC33Hardware.c} erstellt. Die Funktion wurde mit Doxygen dokumentiert (TODO Anhang...).

\newpage
\begin{table}
\begin{tabular}{|l|l|}
	\hline 
	\textbf{Register Bit} & \textbf{Function}\\ 
	\hline
	TRISx 	&determines whether the pin is an input 	or an output\\
			&0:=output, 1:=input\\
			&default: all port pins are defined as inputs after a reset\\
	\hline 
	PORTx 	&read reads the port, write writes the latch \\ 
	\hline 
	LATx 	&read reads the latch, write writes the latch \\ 
	\hline 
	ODCx 	&configures pin for digital or open-drain output \\
			&0:=digital output, 1:=open drain output\\
	\hline 
	CNENx 	&enables change notification (CN) interrupts\\
			&0:=interrupts disabled, 1:=interrupts enabled\\
	\hline 
	CNPUx 	&enables weak pullup\\
			&0:=pullup disabled, 1:=pullup enabled\\  
	\hline 
	CNPDx 	&enables weak pulldown\\
			&0:=pulldown disabled, 1:=pulldown enabled\\  
	\hline 
	ANSELx 	&controls the operation of the analog port pins\\
			&0:=port operates as digital I/O port\\
			&1:=port operates as analog I/O port\\
	\hline
	note 	&Any bit and its associated data and control registers that are not\\
			&valid for a particular device is disabled. This means the correspon\\
			&-ding LATx and TRISx registers and the port pin are read as zeros.\\
	\hline
	note	&The open-drain feature allows the generation of outputs higher\\
			&than VDD (e.g., 5V on a 5V tolerant pin) by using external pull-up\\
			&resistors. The maximum open-drain voltage allowed is the same as \\
			&the maximum VIH specification for that pin.\\
	\hline
	note	&Pull-ups and pull-downs on change notification (CN) pins should\\
			&be disabled when the port pin is configured as a digital output.\\
	\hline
\end{tabular}
\caption{Konfigurationsmöglichkeiten der I/O Ports}
\label{tab:ioports}
\end{table}

\begin{lstlisting}[frame=htrbl, caption={Konfigurieren des Ports RB8 als Digitaler Input mit Pullup Widerstand}, label={lst:confregport}]
//configure RB8 as digit input with pullup
TRISBbits.TRISB8=1;     //configure as input
ANSELBbits.ANSB8=0;     //configure as digital
CNENBbits.CNIEB8=0;     //disable change notification interrupt
CNPUBbits.CNPUB8=1;     //enable weak pullup
CNPDBbits.CNPDB8=0;     //disable weak pulldown
\end{lstlisting}



