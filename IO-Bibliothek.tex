\section{IO-Bibliothek}
\subsection{Aufgabenstellung}
Die populäre Arduino-Plattform (\url{https://www.arduino.cc/en/Reference/}) 
kapselt die Pinkonfiguration und -ansteuerung mit folgenden Funktionen:
\begin{itemize}
\item pinMode()
\item digitalRead()
\item digitalWrite()
\end{itemize}
Übertragen Sie dieses Konzept auf das EDA-Board. Anwendungsbeispiele:
\begin{itemize}
\item pinMode(SW1, IPUT\_PULLUP)
\end{itemize}
soll den Pin, an den SW1 angeschlossen ist, als digitalen Eingang konfigurieren und den Pullup-Widerstand einschalten.
\begin{itemize}
	\item digitalWrite(LED203, HIGH)
\end{itemize}
soll an dem Pin, an den LED203 angeschlossen ist,
einen High-Pegel ausgeben.
Verwenden Sie die Bezeichner aus dem Schaltplan.
Modularisieren Sie Ihre Software, verwenden Sie dazu die Dateinamen \textit{edaPIC33Hardware.h} und \textit{edaPIC33Hardware.c}.\newline \newline
Dokumentieren Sie die Funktionen mit Doxygen.\newline \newline
Messen Sie die Zeit, die zur Ansteuerung eines Ausgangspins mit den IO-Bibliotheksfunktionen notwendig ist
und vergleichen Sie diese mit einem direkten Schreiben in die entsprechenden Hardwareregister.

\subsection{Lösung}
\begin{enumerate}
		\item Lösungsansatz 1
		\item Lösungsansatz 2
		\item Lösungsansatz 3
		\item Lösungsansatz 4
\end{enumerate}


%Beispiel für Quellcode Listening
\begin{lstlisting}[frame=htrbl, caption={Listening Bezeichnung}, label={lst:Referenzname}]
//Quellcode
\end{lstlisting}

