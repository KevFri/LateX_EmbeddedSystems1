\section{Mathematik}

\label{sec:mathematik}
 
\subsection{Unterstufe}
\label{sec:unterstufe}

	\begin{equation*} %stern unterdrückt die nummer
  	a + 2 = c
	\end{equation*}
	\begin{equation*}
  	a_{ij} - a_2 = 0
	\end{equation*}
	\begin{equation*}
  	a_{ij}^2 - a_2 = 0
	\end{equation*} 
	\begin{equation*}
  	\frac{1}{a} + \frac{1}{b} = \frac{a+b}{ab}
	\end{equation*}
	\begin{equation*}
  	\sigma + \tau = \alpha
	\end{equation*}

\subsection{Oberstufe} 
\label{sec:oberstufe}
	\begin{equation}
 	 \label{eq:1}
  	\left( \frac{a}{b} \right)' = \frac{a'b-ab'}{b^{2}}
	\end{equation} 

\subsubsection{Inlinemathematik}
Es gilt die Invariante $b \neq 0$.

\subsubsection{Integrale}
\begin{equation}
  \label{eq:2}
  \int\limits_{a}^{b} x^{2} \, dx = \frac{ b^{3} - a^{3} }{3}
\end{equation}

\subsubsection{Wurzeln}
\begin{equation}
  \label{eq:3}
  c = \sqrt{ a^{2} + b^{2} }
\end{equation}